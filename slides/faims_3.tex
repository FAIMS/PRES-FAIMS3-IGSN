

%----------------------------------------------------------------------------------------
\section{FAIMS 3.0}

\begin{sectionframe} % Custom environment required for section slides
	\frametitle{FAIMS 3.0 Electronic Field Notebooks}
	\framesubtitle{The next generation of FAIMS Mobile}

    \vfill
    
    FAIMS Mobile v2.6 has reached the end of its useful life. In 2020 an investment from the Australian Research Data Commons (ARDC) Platforms program and several partner organisations enabled work to begin on redeveloping FAIMS Mobile using modern components.
% 	This is on another line
\end{sectionframe}

%----------------------------------------------------------------------------------------

\begin{frame}{FAIMS 3.0 redevelopment approach}
Retain FAIMS 2.6 research-specific features (three ‘reference customisations’ incluidng CSIRO chosen) and add:
    \begin{itemize}
        \item Cross-platform (Android, iOS, desktop) functionality
        \item GUI (web application) for customisation (produces definition files)
        \item Preconfigured interoperability with other systems like Cloudstor, EOSC nodes, ELNs, domain repositories including data ‘round trip’ via API or ETL
        \item Better user management and security
        \item A plugin architecture for extensibility
        \item Improved adaptability, scalability, and performance
        \item Expanded device support (cameras, printers, sensors, instruments, etc.)
    \end{itemize}
    \vfill
    Alpha prototype completed June 2021, internal beta release planned December 2021, external beta June 2022, production December 2022.
\end{frame}
%----------------------------------------------------------------------------------------

\begin{comment}


%----------------------------------------------------------------------------------------

\begin{frame}{Where are we now?}
    \begin{itemize}
        \item FAIMS v2.6 is nearing end of its useful life.
        \item The FAIMS team did CSIRO ON Prime in 2016, completing 70+ interviews with clients / potential clients.
        \item A high-level technical plan for FAIMS v3.0 won a US design prize in 2017 \parencite{Bureau_of_Reclamation2017-xl}.
        \item ARDC Platforms announced a major co-investment in late 2019 to rebuild FAIMS using modern components.
    \end{itemize}
\end{frame}
%----------------------------------------------------------------------------------------


\begin{frame}{Opportunities and challenges}
    \begin{itemize}
        \item Useful research-specific features of FAIMS v2.6 to be retained.
        \item Android-only hindered uptake; cross-platform support required.
        \item Lack of self-service customisation / deployment hindered uptake.
        \item Users want to be able to edit data on the desktop or online, then have that data available for further editing in the field (data `round-trip' outside of the application).
        \item Users want more options for accessing / exporting data on-demand.
        \item Users want to be able to use the platform for sensitive data. 
        \item Improved scalability needed (application performance; server-to-server synchronisation)
        \item Enterprise features (orchestration, user management, reporting, branding, SSO, etc.) needed for eventual COSS \textbf{product} to support sustainability (underlying software \textbf{project} will remain OS).
    \end{itemize}
\end{frame}
%----------------------------------------------------------------------------------------

\begin{frame}{FAIMS 3.0 development approach}
Technical elaboration in 2020 produced an \href{https://zenodo.org/record/4616766}{Elaboration Report}; our approach includes:
    \begin{itemize}
        \item NODE.JS
        \item NoSQL datastore (Apache CouchDB / PouchDB)
        \item APIs for data interactions (CouchDB)
        \item Progressive JS Single Page Application wrapped in Native code
for cross-platform support using Capacitor
        \item JSON Forms
        \item Javascript mapping library (OpenLayers)
        \item Plugin architecture

    \end{itemize}
\end{frame}
%----------------------------------------------------------------------------------------


\begin{frame}{FAIMS 3.0 development approach}
Other planned, but not yet elaborated, technical capabilities include:
    \begin{itemize}
        \item Web application to provide GUI to produce definition files
        \item 'Real' user management and security
        \item Interoperability with Cloudstor, EOSC nodes, ELNs, domain repositories (data `round-trip' export-modify-import)
        \item Device support (cameras, printers, instruments, etc.)
        \item Offline mapping
        \item Audio/video management
    \end{itemize}
\end{frame}

%----------------------------------------------------------------------------------------

\begin{frame}
    \frametitle{FAIMS 3.0 development progress}
    \framesubtitle{Development Progress}        
        \begin{itemize}
            \item \href{https://docs.google.com/document/d/13eTN8jhJa3Pgs9GOdo7r4jtIQcskNo7ikxJcBDBKHzw/edit}{Technical Elaboration Report} was approved in February. This established the proof-of-concept for FAIMS3. 
          \item \href{https://github.com/FAIMS/FAIMS3/releases/tag/v0.1.0-alpha}{Alpha prototype} was released on 11 June 2021.  
          \item Alpha prototype passed \href{https://doi.org/10.5281/zenodo.5030772}{user-acceptance testing} on 15 June 2021.
        \item FAIMS3 code has been licensed under the \href{https://www.apache.org/licenses/LICENSE-2.0}{Apache2} license, a Developer Contribution agreement is being applied to all code affirming the license. 
        \item The FAIMS3 repository is now public on \href{https://github.com/FAIMS/FAIMS3}{GitHub}.
        \item FAIMS3 Beta development commenced on 5 July 2021 and will conclude 1 November. 

    \end{itemize}


\end{frame} 

\end{comment}